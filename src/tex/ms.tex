% Define document class
\documentclass[twocolumn]{aastex631}
\usepackage{showyourwork}

% Begin!
\begin{document}

% Title
\title{Gravitational wave parameter estimation at scale: Analyzing the third gravitational wave catalog with Jim}

% Author list
\author{@kazewong}

% Abstract with filler text
\begin{abstract}
Parameter estimation is one of the most commonly performed tasks in gravitational wave data analysis.

\end{abstract}

% Main body with filler text
\section{Introduction}
\label{sec:intro}

\section{Parameter estimation with Jim}

\subsection{Likelihood function}

\subsection{Heterodyne Likelihood}

\subsection{Reparameterization}

\subsection{Adaptive Monte Carlo with normalizing flow}

\subsection{Importance Sampling}

\subsection{Parallel Tempering}

\subsection{Discrete-continuous sampling}

\section{Experiment}

\subsection{Catalog setup}

% Mention waveform models here

\subsection{Special events}

\section{Discussion}

% Say something about the 

% Production at scale

% Reproducible workflow

% 3G deployment

% Mention flexibility and performance

Dingo needs pre-training, which means it will be difficult apply to new models.



\section{Acknowledgements}

\citep{Luger2021}.

\bibliography{bib}

\end{document}
